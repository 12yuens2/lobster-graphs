\section*{Abstract}
The ability to make measurements on lobsters is important in monitoring the size and health of the creatures.
This project explores the representation of lobsters as attributed graphs to discover and measure properties such as their size, using an existing dataset of lobster images.
Computer vision techniques are applied to detect interest points of different lobster body parts. 
Probabilistic models and subgraph matching are then used to label and build up the graph representation.
The advantages and disadvantages of the various methods and algorithms used are discussed before the final methods were chosen. 
The proposed approach can be used to replace existing manual methods of measurement to improve the efficiency of both fishers and scientists in the field. 
Positive results were acquired for aspects of detection and identification, which shows the viability of the graph matching approach.
The effectiveness of the methods are evaluated against the existing dataset to discover weaknesses in using graph matching as a technique for lobster categorisation. 
%The aim is to use these techniques to discover properties of the lobster, for example its age, gender and health.  TODO results

%Extensions to this project would be to develop a new algorithm for create graphs rather than using existing ones and to try the same techniques on more complex images with lobsters in their natural environment.

