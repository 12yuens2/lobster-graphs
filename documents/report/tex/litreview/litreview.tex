\section{Context Survey}
\renewcommand{\imgpath}{tex/litreview/imgs}

\subsection{Background}

\subsubsection{Previous work}
A large part of this project follows on from \cite{lobster-thesis}. In his work, Abdallah created a dataset consisting of images and features of lobsters. The lobsters were measured and categorised and segmentation and feature extraction techniques were applied to create a more diverse dataset with baseline results.  Additionally, classification and regression techniques were used to both classify the category of the lobster (juvenile or mature) and predict the carapace length. 
\n
This project is a continuation of Abdallah's work to apply computer vision techniques

\subsection{Related work}

\subsubsection{Human pose recognition}
The use of graphs in human pose estimation has been studied in the past
\cite{human-pose} \cite{human-skeleton} where labelled nodes that represent important features such as hands and head are used to build skeleton models.
\n
Some of the methodology used in human pose estimation can be applied to our lobster matching problem. In particular, the paper on human pose estimation using a topological graph database \cite{human-pose} by Tanaka et al.  also uses a graph matching technique with an attributed graph to match skeletons to corresponding human postures. In their paper, example skeletons with different human topologies are developed into attributed graphs with manually assigned body part labels and stored in a model graph database. Any input skeletons can then also be converted to an attributed graph and matches with the examples in the database. Because their method of subgraph matching produces multiple results, further filters were needed to reduce the remaining graphs to one correct match. 
\imagefig{0.6\textwidth}{\imgpath/human-pose.png}{Body part identification algorithm using a model graph database (MGDB). Extracted from \cite{human-pose}.}
\noindent
The methodology in this project, which is explained in section \ref{sec:design}, follows some parts of Tanaka's method. Most notably the use of example lobster graphs stored in a database for input subgraphs to match to. The aspect of having labelled attributed graphs of human skeletons extends well to labelled graphs of lobster. 
 
\subsubsection{Cattle identification}
The concept of combining local invariant features (keypoints) and graph matching has also been studied for use in automatic cattle identification \cite{cattle}. 

\subsection{Graph matching problem}
The graph matching, or subgraph isomorphism problem is where given two undirected graphs $G_1$ and $G_2$, it must be determined whether the graph $G_1$ contains a subgraph that is isomorphic to $G_2$ \cite{subgraph}. Cook showed in his paper that subgraph isomorphism was NP-complete with a reduction to the 3-SAT problem. 
\n
This problem applies to the project from the use of subgraph matching to determine if a labelled subgraph is contained in a larger complete graph of a lobster. 
\n
In this project, the problem of graph matching extends beyond that of pure subgraph isomorphism, where only the number of vertices and its connections are relevant. Because the shape of the lobster is a crucial part in determining successful matches, the size (weight) of each node in the graph and the length (weight) of the edges are all included as an important aspect in matching. 